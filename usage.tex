\section{Usage Scenarios}
\label{sec:scenarios}

Quantative consistency metrics are useful for a variety of services
that can tolerate eventual consistency. In this section, we outline
three use cases in the context of a hypothetical microblogging web
service, Twissistency.

\subsection{Dynamic Requests}

\textit{Twissistency's data scientists have learned that their users
  respond negatively to high latencies, and the company sets a desired
  latency bound of 50ms for their back-end data store requests.}\\

Data stores often offer a choice between replication parameters. For
example, in Dynamo-style sytems, applications can choose the minimum
number of replicas to read from ($R$) and write to ($W$). If $R$+$W$
is greater than the number of replicas ($N$), applications will read
their writes, and, if not, the system provides eventual
consistency. With a replication factor of $N=3$, we have three options
for eventually consistent operation: $R$$=$$W$$=$$1$, $R$$=$$1$$,
W$$=$$2$, and $R$$=$$2$$, W$$=$$1$. $R$$=$$W$$=$$1$ is guaranteed to
be fastest, but it is also the least consistent.

\\x\textit{Based on feedback from their infrastructure team, the
  application developers set $R$=$W$=$1$ for their Dynamo-based data
  store. This meets the latency SLA with an average latency of 10ms,
  but they notice that many requests return stale data. They would
  like to use $R$=$W$=$2$, but this is too slow, while other
  eventually consistent operations are too slow only during periods of
  peak traffic.}\\

\subsection{DBA Integration}

Replication parameter

Read repair rate

Slow nodes

\subsection{Monitoring and Alerts}

Devops get paged in the middle of the night
