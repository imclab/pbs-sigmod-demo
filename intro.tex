\section{Introduction}

Modern distributed data stores offer a choice. Weak consistency models
are fast and guarantee ``always-on'' behavior, but provide limited
semantics. Stronger consistency models are easier to reason about but
slower and potentially unavailable. The choice of consistency has
wide-ranging implications for application writers, operations
management, and end-users, and, in light of its performance benefits,
weak consistency is often acceptable.

Using one predominant consistency model, eventual consistency, is, on
its face, a cavalier proposition. Prevalent models such as eventual
consistency provide no guarantees as to when new writes will become
visible to readers and what versions of data items will be presented
in the interim. Reading all \texttt{null} values from a database is
eventually consistent. The data store can delay write visbility for
decades and not violate its guarantees. Yet, despite these weak
semantics, there is common sentiment among practitioners that eventual
consistency is often ``good enough'' and ``worthwhile'' for many
applications.

In recent work, we provided an analytical and empirical framework for
analyzing the consistency provided by eventually consistent stores, or
``how eventual and how consistent is eventual consistency?'', called
Probabilistically Bounded Staleness, or PBS. Eventually consistent
stores do not make promises about answers to the length of time
required to observe an update or the staleness of values, but this
does not preclude us from making informed statements about what is
likely to happen. By using expert knowledge of the underlying data
store and its replication protocols in addition to some lightweight
in-situ profiling, we can inform data store users about what
consistency they are likely to observe in the future.

Our PBS predictions are part of a larger trend towards providing
quantitative measurements and analysis of weakly consistent
stores. There is recent work ranging from distributed systems theory
to database practitioner communities on measuring violations of
properties such as atomicity, serializability, and regular register
semantics. PBS in particular has experienced relative popularity in
the NoSQL community and our implementation of PBS for quorum-based
systems was recently accepted to Cassandra's mainline source trunk and
will be released by the end of 2012. Afeter discussions with other
researchers pursuing consistency monitoring and verification, we
expect more technology transfer in this space.

While monitoring and prediction are useful in themselves, perhaps more
importantly, they enable a rich space of applications that we believe
have been neglected. For example, in our initial PBS work, we mostly
assume that simply recording the predictions is a worthwhile task:
what they are used for is mostly left for future work. We are most
excited about three particular applications: \textit{i.)} dynamic
request-based consistency parameters, or auto-tuning request routing
based on a given latency and consistency service level agreement,
\textit{ii.)} database administration tasks with respect to the impact
of slow nodes and networks, replication factors, and data store
paramers like anti-entropy rates, and \textit{iii.)} integration into
traditional alerts and monitoring frameworks. We elaborate further in
Section~\ref{sec:scenarios}, but, in short, these quantitative metrics
allow new ways to tune the performance, semantics, and the maintenance
of eventually consistent stores.

In this demo proposal, we outline these advanced use cases in detail
(Section~\ref{sec:scenarios}), describe how both quantitative metrics
can be integrated into existing stores and architectures based on our
experiences with the Cassandra community
(Section~\ref{sec:architecture}), and sketch how we plan to
demonstrate this newly enabled functionality, allowing SIGMOD
attendees to act as both end-users and operations managers for an
eventually consistent service (Section~\ref{sec:demo}).
